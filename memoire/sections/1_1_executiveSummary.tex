The work undertaken during those 6 months is a continuation of my previous studies in automated detection of Archaelogical targets in Geophysical Surveys. 

Here, we use state of the art Deep Learning techniques on \gls{lidar} surveys to detect 3 classes of targets: barrows, celtic fields and charcoal kilns, whose description and definition can be found in Section~\ref{elemArchaeo}. The dataset used are \gls{lidar} surveys that have been done in a large region of the Netherlands, and which has already been used in a number of automated detection tasks. The goal here is to evaluate the performance of a modified YOLO\cite{yolov4} in this particular kind of object detection task. We will use techniques taken from detection using satellite imagery, as they often deal with issues similar to detection in large \gls{lidar} surveys. 

Automated object detection is a highly studied subdomain of Deep Learning, which has made very important advances in the recent year and performance of the models has skyrocketed, along with their speed. Object detection, and more generally Computer Vision is in very high demand, but most research is done on what we call "natural images", i.e. photographs, and not much is done using geophysical surveys. There is a need for high performance and fast object detection systems using different sensors in multiple fields. 

Le travail entrepris durant ces 6 mois est dans la continuité de mes précédentes recherche dans la détection automatique de cibles Archéologique dans des relevés géophysique. 

Ici, nous utilisons des techniques de pointe de Deep Learning dans des relevés \gls{lidar} pour détecter 3 classes de cibles: des tumulus, des champs Celtiques et des charbonnières, dont la description et la définition peut être trouvé en Section~\ref{elemArchaeo}. Le dataset utilisé ici est un ensemble de relevés \gls{lidar} ayant été faite dans une grande région des Pays-Bas, et qui a déjà été utilisés dans un certain nombre de projet de détection automatique d'objet archéologique. Le but ici est d'évaluer la performance d'un modèle YOLOv4\cite{yolov4} optimisé pour ce genre de tâche. Nous allons également faire usage de techniques empruntés a la détection dans des images satellite, qui partagent certaines caractéristiques et problématiques. 

La détection automatique d'objet est un sous domaine très étudié du Deep Learning et de la vision automatique et qui a connus de très grandes avancées dans les années précédentes. La performance, à la fois en terme de précision mais aussi de vitesse d'exécution a bondi en avant. La détection d'objet, et plus généralement la Vision Automatique est très demandé actuellement, mais la majeure partie des études et modèle se font sur ce que nous appelons des "images naturelles", i.e. des photographies, et peu en comparaison sur des relevés géophysique. Il y a un besoin dans plusieurs domaines pour des détecteurs d'objet rapide et précis étant capable d'analyser des images fondées sur d'autre signaux que le spectre visible.
