The University of Leiden holds the fifth place in the SustainaBul, a ranking of the most sustainable higher educations institutions. This ranking is done by the national student network in the Netherlands, the \textit{Studenten Voor Morgen}. Industrial Ecology and biodiversity, among other sustainable themes are an important part of the University research and teachings. 

Leiden University also holds the 7th place in UI Green Metric, a sustainability ranking created by the University of Indonesia. This rankings looks into a given University take on the challenges of energy, recycling, transportation, water or climate change. The University of Leiden quickly rose among this ranking, coming from the 49th place only two years ago. This rise in ranking comes from the multiple actions undertaken by the University to create a more sustainable and carbon neutral environment. For example, many extra solar panels were installed on the University buildings, and the waste generated is now recycled in the entire University. Working with other universities through their national network has also helped Leiden to gather and share much more information to the organisers of the UI Green Metric. Effort is still ongoing to enhance the sustainability score by investing more funds in the areas of Education, Research and into greener infrastructure. The National goal is to make all of Netherlands Universities enter the ranking top 50. 

Leiden University publish its policies and efforts towards making the institution more sustainable each year with an annual sustainability report. Much progress has been made in the last year. The entire vehicle fleet of the University is now electric and its restaurants and cafés have almost entirely eliminated beef products. Much effort have been directed into reducing the institution CO2 footprint, being more economical with ressources and reducing energy consumption. The University also compensates for its CO2 emissions by financially supporting sustainable projects who aims at reducing the CO2 of the world via green certificates. The University of Leiden has decreased its carbon footprint by 51\% compared to 2016. The Leiden University Green Office (LUGO) is a student lead system of checks and balance that aims at keeping the University true to its commitment to sustainability.  


