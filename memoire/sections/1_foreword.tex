This internship was not as traditional as it could have been: both because of the COVID-19 Situation, which profoundly modified the course of my work, but also because it was at its heart a research position, and not a more traditional developper/manager type of work. While this type of work is not typically assumed by engineers in their final year of study, this experience has proven to me to be unvaluable, as it has allowed me to delve deep into research I have been doing in my free time; and gave me the means and the time to create what was only a dream for me a few years ago: a dedicated object detection model optimized for Archaelogical Remote Sensing Surveys. 

As such, this document is also not traditional, and is not be a simple insternship report. It is much more detailed in the work undertaken, the previous state of the art, be it in automated detection in archaeology or in satellite imagery. This project is a continuation of work I have started almost 3 years ago, and was my first opportunity to dedicate an entire section of my life working full time towards it, which is invaluable. While I believe this project is a success, because it achieved its stated goal, there is much work left to be done. Digital Archaeology is flourishing, and there is vast amounts of data, be it digital like geophysical surveys or manuscript like digging reports, that are still untouched. I truly believe that this treasure trove, combined with advanced machine learning techniques might gives novel way to understand our past. By doing so, we might be better prepared for the challenges that are coming to us.

I can only give my deepest and most sincere thanks to the Leiden University and both Professor Lambers and Mr. Verschoof-van der Vaart MA for their help. This institution accepted my offer and gave me an opportunity to push myself, which I will stay forever grateful for. 


