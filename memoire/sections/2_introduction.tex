This document is served as the final assignment of my graduate engineering degree which concludes my 5 years as an ESIEA Student.  This document details my work during my 6 months stay as an intern and research student at Leiden University.

This internship was done at the department of Digital Archaeology, which aims to apply information technology to Archaeology. Leiden University is one of the top institution in the world, and this department is to my knowledge the best in the subfield of automated detection. This work was done under the supervision of Mr. Verschoof-van der Vaart, a PhD candidate whose work has heavily influenced mine. Mr. Verschoof-van der Vaart also studies detection methods. In his 2019 article, Mr. Verschoof-van der Vaart introduces a new kind of dataset based on \gls{lidar} surveys in a central region of the Netherlands, which was used here. He also gave me some very interesting techniques and metrics, and pushed me to think more deeply about the work I was undertaking. 

In short, the aim of this project is to create a dedicated object detection framework for use in Archaeological detection in \gls{lidar} surveys. We use novel methods coming from satellite imagery detection, along with the state of the art in detection techniques to accomplish this task. As such, this project is significantly oriented towards research and development. A fairly comprehensive State of the Art has been redacted, along with a complete document detailing the architecture of the network, the type of tricks and tools used, along with training and testing methodology. Finally some discussion on our results, along with hints on the direction of further research are given. 

 While this work could seem somehow academic and as such "hors sol", some very interesting insights can be glanced from working with Archaeology and Archaeologist. In its essence, Archaeology deals with the incomplete, the unknowable and the fuzziness of reality; things that are also main concerns with Engineering. This is a common theme that we will find throughout this document: given imperfect tools and measures of reality, how can we develop an understanding and give some forms of guarantees on the performance and accuracy of our methods ?  \textbf{Understanding how Archaeologists deal with this uncertainty and the practical aspects of their subject can give us keys in doing a better work in our own field.} As such, I feel that there is a close relation between Engineering and Archaeology, and that more cross field projects need to be done to more firmly tie those two domains together.
